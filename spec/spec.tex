\documentclass[11pt]{report}

\usepackage{fullpage}
\usepackage{fancyhdr}
\setlength{\headheight}{15.2pt}
\setlength{\headsep}{10.2pt}
\pagestyle{fancy}
\lhead{Construct: Programming with Geometry}
\rhead{Sam Gruber}

\usepackage{tabularx}
\usepackage{graphicx}

\begin{document}

\chapter{The Structure of Programs}
\label{chap:struct}

A program in Construct is a Definition (see Section~\ref{sec:def}) which follows a well-defined procedure to generate {\it final} objects from {\it initial} objects. 
Since a program is a Definition, it can used in other programs to asbtract away higher-level operations used for generating more complex objects.

A program is a tuple:

$$(O, D, I, F)$$

$O$ is the set of all objects (see Section \ref{sec:obj}) that are present in the program. 
$D$ is a function $\rho \to \{\delta_1, \delta_2, \dots\}$ which, given an object $\rho$, produces the set of all definitions (see Section \ref{sec:def}) which describe $\rho$. 
$I$ is the set of all initial objects. 
$F$ is the set of all final objects.

Alternatively, $O$ and $D$ may be combined to generate a directed acyclic graph $G$. 
Any valid topological sorting of $G$ is a valid ordering of runtime execution for the program.

\section{Objects}
\label{sec:obj}

An object in Construct is a named entity $\rho$ which exists in 2d space. 
Objects may have position, orientation, magnitude, or any other characteristics which are appropriate based upon their Type (see Section \ref{sec:types}). 

At any point in the execution of a Construct program, any object is a representative of the {\it valid parameter set}. 
This is a set which contains all parameter combinations which are valid based upon the Definitions (see Section \ref{sec:def}) of the object.

\section{Types}
\label{sec:types}

Objects in Construct are classified into one of several types:\\

\noindent\begin{tabularx}{\textwidth}{l l l X}
$\tau$ & $::=$ & $\mathsf{pt}$ & A point in 2d space. \\
 & & $\mathsf{ln}$ & An infinite line in 2d space. \\
 & & $\mathsf{circ}$ & A circle in 2d space. \\
 & & $\mathsf{ang}$ & An angle in 2d space. \\
 & & $\mathsf{dist}$ & A distance in 2d space. \\
 & & $\mathsf{set}(\tau)$ & A homogenous collection of objects of type $\tau$. \\
 & & $\mathsf{grp}(\tau_1, \dots, \tau_n)$ & A ordered collection of objects.
\end{tabularx}

Additionally, some built-in Definitions (see Section \ref{sec:def}) accept numerals as parameters. 
Numerals are shown here with a bar, for example $\bar{n}$ is some integer $n$.
These numerals allow convenient specification of constant values. 
Programmer-created definitions can not accept numerals as initials. 

\section{Definitions}
\label{sec:def}

Definitions are rules that are used to describe an object based upon other objects. All of an object's definitions are evaluated at runtime, during Reconciliation (see Section~\ref{sec:reconc}). 

Each Definition is a member of a Definition Class $\delta_\tau$, which is the class of all Definitions that describe objects of type $\tau$. 
Each Definition's inputs and outputs must be a single object, though groups and sets may be passed.
A Definition may be totally described in this notation (group syntax simplified for brevity): 

$$\mathsf{defname}[\tau](\rho_1 : \tau_1; \rho_2 : \tau_2; \dots)$$

If the Definition is listed as part of a Definition Class (as below), we omit the bracketed type information for brevity. Additionally, we do not specify the type in our syntax (see Chapter \ref{chap:howto}) because it can be derived from context.

\subsection{Generic Definitions}
\label{subsec:def-gen}

The following Definitions describe objects of any type $\tau$. \\

\noindent \begin{tabularx}{\textwidth}{l l l X}
$\delta_{\tau}$ & $::=$ & $\mathsf{id}(\rho : \tau)$ & This object is identical to $\rho$. \\
 & & $\mathsf{of}(s : \mathsf{set}(\mathsf{\tau}))$ & This object is identical to a member of the set $s$. \\
 & & $\mathsf{elem}(\bar{m}; \; g : \mathsf{group}(\tau_1, \dots, \tau_n))$ & This object is the $m$th object in the group $g$. If $m > n$ or $\tau_m \neq \tau$, raise $\mathsf{DefErr}$.
\end{tabularx}

\subsection{Definitions for $\mathsf{pt}$ Objects}
\label{subsec:def-pt}

\begin{tabularx}{\textwidth}{l l l X}
$\delta_{\mathsf{pt}}$ & $::=$ & $\mathsf{on}(l : \mathsf{ln})$ & This point occurs somewhere along $l$. \\
 & & $\mathsf{on}(c : \mathsf{circ})$ & This point occurs somewhere along $c$. \\
 & & $\mathsf{opp}(p : \mathsf{pt}; \; l : \mathsf{ln})$ & This point exists on the opposite side of $l$ from $p$. \\
 & & $\mathsf{center}(c : \mathsf{circ})$ & This point is at the centerpoint of $c$. \\
 & & $\mathsf{inside}(c : \mathsf{circ})$ & This point occurs inside of $c$. \\
 & & $\mathsf{to}(p : \mathsf{pt}; \; d : \mathsf{dist})$ & This point is at distance $d$ from $p$. \\
 & & $\mathsf{to}(l : \mathsf{ln}; \; d : \mathsf{dist})$ & This point is at distance $d$ from the nearest location on $l$. \\
 & & $\mathsf{to}(c : \mathsf{circ}; \; d : \mathsf{dist})$ & This point is at distance $d$ from the nearest location on the circumference of circle $c$. \\
\end{tabularx}

\subsection{Definitions for $\mathsf{ln}$ Objects}
\label{subsec:def-ln}

\begin{tabularx}{\textwidth}{l l l X}
$\delta_{\mathsf{ln}}$ & $::=$ & $\mathsf{thru}(p : \mathsf{pt})$ & This line passes through $p$. \\
 & & $\mathsf{intr}(l : \mathsf{ln})$ & This line intersects $l$ at some location. \\
 & & $\mathsf{par}(l : \mathsf{ln})$ & This line is parallel to $l$. \\
 & & $\mathsf{perp}(l : \mathsf{ln})$ & This line is perpendicular to $l$. \\
 & & $\mathsf{skew}(a : \mathsf{ang}; \; l : \mathsf{ln})$ & This line is at angle $a$ to line $l$. \\
 & & $\mathsf{tan}(c : \mathsf{circ})$ & This line has a point of tangency to $c$ at some location. \\
 & & $\mathsf{cross}(c : \mathsf{circ})$ & This line intersects $c$ at two locations. \\
 & & $\mathsf{to}(p : \mathsf{pt}; \; d : \mathsf{dist})$ & This line is at distance $d$ from point $p$ at it closest location. \\
 & & $\mathsf{to}(l : \mathsf{ln}; \; d : \mathsf{dist})$ & This line is at distance $d$ to line $l$ at their closest points. In two dimensions, this implies $\mathsf{par}(l)$. \\
 & & $\mathsf{to}(c : \mathsf{circ}; \; d : \mathsf{dist})$ & This line is at distance $d$ from the nearest location on the circumference of circle $c$. \\
\end{tabularx}

\subsection{Definitions for $\mathsf{circ}$ Objects}
\label{subsec:def-circ}

\begin{tabularx}{\textwidth}{l l l X}
$\delta_{\mathsf{circ}}$ & $::=$ & $\mathsf{thru}(p : \mathsf{pt})$ & This circle passes through $p$. \\
 & & $\mathsf{about}(p : \mathsf{circ})$ & This circle is centered on $p$. \\
 & & $\mathsf{tan}(l : \mathsf{ln})$ & This circle has a point of tangency to $l$ at some location. \\
 & & $\mathsf{tan}(c : \mathsf{circ})$ & This circle has a point of tangency to $c$ at some location. \\
 & & $\mathsf{cross}(l : \mathsf{ln})$ & This circle intersects $l$ at two locations. \\
 & & $\mathsf{to}(p : \mathsf{pt}; \; d : \mathsf{dist})$ & This circle's circumference is at distance $d$ from point $p$ at it closest location. \\
 & & $\mathsf{to}(l : \mathsf{ln}; \; d : \mathsf{dist})$ & This circle's circumference is at distance $d$ to line $l$ at their closest points. \\
 & & $\mathsf{to}(c : \mathsf{circ}; \; d : \mathsf{dist})$ & This circle's circumference is at distance $d$ from the nearest location on the circumference of circle $c$. \\
\end{tabularx}

\subsection{Definitions for $\mathsf{ang}$ Objects}
\label{subsec:def-ang}

\begin{tabularx}{\textwidth}{l l l X}
$\delta_{\mathsf{ang}}$ & $::=$ & $\mathsf{join}(a : \mathsf{ang}; \; b : \mathsf{ang})$ & This angle's sweep is equal to the sum of the sweeps of $a$ and $b$. \\
 & & $\mathsf{split}(a : \mathsf{ang}; \; \bar{n})$ & This angle's sweep is equal to $\frac{1}{n}$ times the sweep of $a$. If $n = 0$, raise $\mathsf{DefErr}$. \\
 & & $\mathsf{sweep}(\bar{n})$ & The sweep of this angle is equal to $n$ (in degrees). \\
 & & $\mathsf{between}(k : \mathsf{ln}; \; l : \mathsf{ln})$ & This angle's sweep is the sweep between the lines $k$ and $l$. \\
\end{tabularx}

\subsection{Definitions for $\mathsf{dist}$ Objects}
\label{subsec:def-dist}

\begin{tabularx}{\textwidth}{l l l X}
$\delta_{\mathsf{dist}}$ & $::=$ & $\mathsf{sum}(d : \mathsf{dist}; \; t : \mathsf{dist})$ & This distance is equal to the sum of the lengths of $d$ and $t$. \\
 & & $\mathsf{div}(d : \mathsf{dist}; \; \bar{n})$ & This distance is equal to $\frac{1}{n}$ times the length of $d$. If $n = 0$, raise $\mathsf{DefErr}$. \\
 & & $\mathsf{length}(\bar{n})$ & This distance is equal to $n$.\\
 & & $\mathsf{span}(p : \mathsf{pt}; \; q : \mathsf{pt})$ & This distance is equal to the distance from $p$ to $q$. \\
 & & $\mathsf{span}(p : \mathsf{pt}; \; l : \mathsf{ln})$ & This distance is equal to the shortest distance from $p$ to $l$. \\
 & & $\mathsf{span}(p : \mathsf{pt}; \; c : \mathsf{circ})$ & This distance is equal to the shortest distance from $p$ to $c$. \\
 & & $\mathsf{span}(k : \mathsf{ln}; \; l : \mathsf{ln})$ & This distance is equal to the shortest distance from $k$ to $l$. \\
 & & $\mathsf{span}(l : \mathsf{ln}; \; c : \mathsf{circ})$ & This distance is equal to the shortest distance from $l$ to $c$. \\
 & & $\mathsf{span}(b : \mathsf{circ}; \; c : \mathsf{circ})$ & This distance is equal to the shortest distance from $b$ to $c$. \\

\end{tabularx}

\subsection{Definitions for $\mathsf{set}(\tau)$ Objects}
\label{subsec:def-set}

\begin{tabularx}{\textwidth}{l l l X}
$\delta_{\mathsf{set}}$ & $::=$ & $\mathsf{empty}$ & This set has no members. \\
 & & $\mathsf{size}(\bar{n})$ & The number of members of this set. \\
 & & $\mathsf{include}(\rho : \tau; s : \mathsf{set}(\tau))$ & This set contains all members of the set $s$ and additionally contains $\rho$. \\
 & & $\mathsf{exclude}(\rho : \tau; s : \mathsf{set}(\tau))$ & This set contains all members of the set $s$ except $\rho$. If $\rho$ is not a member of $s$, raise $\mathsf{DefErr}$.
\end{tabularx}

\subsection{Definitions for $\mathsf{grp}(\tau_1, \dots , \tau_n)$ Objects}
\label{subsec:def-grp}

\begin{tabularx}{\textwidth}{l l l X}
$\delta_{\mathsf{grp}}$ & $::=$ & $\mathsf{collect}(\rho_1 : \tau_1, \dots, \rho_n : \tau_n)$ & This group contains $\rho_1, \dots, \rho_n$.
\end{tabularx}


\section{Modifiers}
\label{sec:mods}

Modifiers are higher-level constructs than Definitions which further specify program behavior.

\subsection{Modifiers for Objects}
\label{subsec:mods-obj}

\begin{tabularx}{\textwidth}{l l l X}
$\mu_{\rho}$ & $::=$ & $\mathsf{initial}$ & $\rho$ is given as an input to the program. \\
 & & $\mathsf{final}$ & $\rho$ is the result of the program. \\
 & & $\mathsf{unique}$ & The Definitions which specify $\rho$ must resolve to exactly one possible set of parameters. If $\rho$ is not uniquely defined, raise $\mathsf{DefErr}$.
\end{tabularx}

\subsection{Modifiers for Definitions}
\label{subsec:mods-defs}

\begin{tabularx}{\textwidth}{l l l X}
$\mu_{\delta_\tau}$ & $::=$ & $\mathsf{not}$ & The negation of this Definition. \\
 & & $\mathsf{handle}(\delta_\tau')$ & If this Definition raises a runtime error, instead apply $\delta_\tau'$. \\
 & & $\mathsf{each}$ & Applies this Definition over all members of a set, specifying a new Definition in the class $\delta_{\mathsf{set}(\tau)}$. If $\mathsf{DefErr}$ is raised at any point, this whole definition raises $\mathsf{DefErr}$. \\
 & & $\mathsf{filter}$ & Applies this Definition over all members of a set. If the definition does not raise an error, include its {\it final} object in the new set. Otherwise, skip that object without raising $\mathsf{DefErr}$. \\
\end{tabularx}

\subsection{Modifiers for Graphs}
\label{subsec:mods-graphs}
\begin{tabularx}{\textwidth}{l l l X}
$\mu_{G}$ & $::=$ & $\mathsf{local}$ & Produces a local definition from this graph. \\
\end{tabularx}


\chapter{Execution}
\label{chap:exec}

\section{Runtime Errors}
\label{sec:runtime-errors}

\begin{tabularx}{\textwidth}{l l l X}
$\epsilon$ & $::=$ & $\mathsf{DefErr}$ & A set of Definitions (see Section \ref{sec:def}) could not be Reconciled (see Section \ref{sec:reconc}) to a valid object.
\end{tabularx}

\section{Reconciliation}
\label{sec:reconc}

When an object in the Evaluation DAG becomes definable (when all of its immediate prerequisites are defined), the Construct runtime begins a process of Reconciliation. 
During Reconciliation the runtime attempts to find a state for the object which satisfies all of the Definitions applied to it. 
If this is not possible, the runtime declares $\mathsf{DefErr}$.

\chapter{How to Program in Construct}
\label{chap:howto}

\section{Example Programs}
\label{sec:example}

\subsection{Triangle Area}
\label{subsec:triarea}

\noindent \begin{tabularx}{\textwidth}{l X p{4cm}}
\raisebox{-.5\height}{\includegraphics{../triarea/stage1.pdf}} & {\tt initial grp(pt;pt;pt) tri \newline
        pt p1.elem(1,tri) \newline
        pt p2.elem(2,tri) \newline
        pt p3.elem(3,tri) } & {\small Start with three points that describe the triangle.} \\
\hline
\raisebox{-.5\height}{\includegraphics{../triarea/stage2.pdf}} & {\tt ln l1.thru(p1).thru(p2)} & {\small Add a line along one of the triangle's sides.} \\
\hline
\raisebox{-.5\height}{\includegraphics{../triarea/stage3.pdf}} & {\tt dist two.length(2) \newline
        ln l2.thru(p1).thru(p3) \newline
        pt p4.on(l1).to(p1;two).opp(p2;l2)} & {\small Create a new point 2 units outside the triangle along the line.} \\
\hline
\raisebox{-.5\height}{\includegraphics{../triarea/stage4.pdf}} & {\tt ln l3.thru(p3).thru(p4) \newline
        ln l4.thru(p2).par(l3) \newline
        pt p5.on(l1).on(l4)} & {\small The triangle (p1,p4,p5) has the same area as the original triangle.} \\
\hline
\raisebox{-.5\height}{\includegraphics{../triarea/stage5.pdf}} & {\tt ln l5.thru(p5).par(l1) \newline
        ln l6.thru(p4).perp(l1) \newline
        pt p6.on(l5).on(l6)} & {\small The triangle (p1,p4,p6) has the same area as the original triangle.} \\
\hline
\raisebox{-.5\height}{\includegraphics{../triarea/stage6.pdf}} & {\tt final dist area.span(p4;p6)} & {\small This distance is the area because the base of the triangle is 2, so the height equals the area.}
\end{tabularx}

\subsection{Adjacent Segments}
\label{subsec:adjsegs}

This program demonstrates a verification definition. Unlike other definitions which produce a resulting object, this definition just ensures that some relationships are satisfied over its inputs. Verifications may be used as part of extracting objects from sets, or as contracts in programs.

Here we use an abbreviated textual syntax for referencing objects in groups. {\tt obj[1][2]} references the second item in a group which is the first item of a group object named {\tt obj}. This notation is similar to array addressing in modern textual programming langauges. \\

\noindent \begin{tabularx}{\textwidth}{l X p{4cm}}
\hspace{3.5cm}\vspace{1cm} & {\tt initial grp(grp(ln;circ);\newline grp(ln;circ)) segs } & {\small Start with two segments (represented by a line and a bounding circle). } \\
\hline
\vspace{0.5cm} & 
{\tt
pt adj.on(segs[1][1]) \newline
      .on(segs[1][2]) \newline
      .on(segs[2][1]) \newline
      .on(segs[2][2])
} & {\small Verify that the lines and circles all intersect at a common point.} \\
\end{tabularx}\\\\

If this simple definition fails, then we know that our line segments are not adjacent.

\subsection{Reduce Polygon to Triangle}
\label{subsec:reducepoly}

TODO

\subsection{Polygon Area}
\label{subsec:polyarea}

The next example program incorporates the previous examples and calls them as user-created definitions. We use the keyword {\tt void} to where a verification definition produces no resulting object; in the geometric interface, this is invisible to the programmer.

User-created definitions are enclosed in curly braces to distinguish them from the built-in definitions. Local definitions are also enclosed in curly braces, but not named. \\

\noindent \begin{tabularx}{\textwidth}{l X p{4cm}}
\hspace{3.5cm} & {\tt final set(grp(ln;circ)) tri.local\{ \newline
  initial set(grp(ln;circ)) poly.size(3) \newline
  void.\{Closed Polygon\}(poly) \newline
  final poly \newline
\}} & {\small If the polygon has three sides, then it's already a triangle. } \\
\hline
 & {\tt handle( local\{ \newline
  initial set(grp(ln;circ)) poly \newline
  void.\{Closed Polygon\}(poly) \newline
  final poly3.\{Reduce Polygon to \newline Triangle\}(poly) \newline
\})} & {\small If the polygon has more than three sides, we reduce it to three. } \\
\hline
 & {\tt grp(ln;circ) A.of(tri) \newline
        grp(ln;circ) B.of(tri) \newline
        .not id(A) \newline
        grp(ln;circ) C.of(tri) \newline
        .not id(A) \newline
        .not id(B) } & {\small Get the three sides. } \\
\hline
\vspace{1.5cm} & {\tt pt p1.on(A[1]).on(B[1]) \newline
        pt p2.on(A[1]).on(C[1]) \newline
        pt p3.on(B[1]).on(C[1])} & {\small Get the three vertices. } \\
\hline
\vspace{1.5cm} & {\tt grp(pt;pt;pt) verts.collect(p1;p2;p3) \newline
        final dist area.\{Triangle Area\}(verts)} & {\small Use the triangle area definition. } \\
\end{tabularx}\\\\

% \section{Interfaces}
% \label{sec:data}

% \subsection{Geometry Markup}
% \label{sec:gml}

% A file format is proposed to allow description of well-defined objects in 2d space. 
% The Construct runtime makes use of these files for input and output.

\end{document}